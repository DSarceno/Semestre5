\documentclass[conference]{IEEEtran}
\IEEEoverridecommandlockouts
% The preceding line is only needed to identify funding in the first footnote. If that is unneeded, please comment it out.
\usepackage{amsmath,amsthm,amssymb} %modos matemáticos y  simbolos
\usepackage{latexsym,amsfonts} %simbolos matematicos
\usepackage{cancel} %hacer la linea que cancela las ecuaciones
\usepackage[spanish, es-noshorthands]{babel} %comandos en español y cambia el cuadro por la tabla
\decimalpoint %cambia las comas por puntos decimal
\usepackage[utf8]{inputenc} %caracteristicas del español
\usepackage{physics} %Simbolos fisicos
\usepackage{array} %mejores formatos de tabla
\parindent =0cm %sangria 
\usepackage{algorithmic}
\usepackage{graphicx}
\usepackage{textcomp}
\usepackage{xcolor}
\usepackage{mathtools} 
\usepackage[framemethod=TikZ]{mdframed}%Entornos talegas
\usepackage[colorlinks = true,
			linkcolor = blue,
			citecolor = black,
			urlcolor = blue]{hyperref}%formato de los links y URL's
\usepackage{multicol} %varias columnas
\usepackage{enumerate} %enumeraciones
\usepackage{pgf,tikz,pgfplots} %documentos en formato tikz
\usepackage{mathrsfs} %letras chingonas (transformada de laplace)
\usepackage{subfigure} %varias figuras seguidas
\usepackage{tabulary}
\usepackage{multirow} %ocupar varias filas en una tabla
\usepackage{fancybox} %recuadros talegas
\usepackage{float} %ubicar graficas
\usepackage{color}
\usepackage{comment}
\usepackage{stackrel}
\usepackage{calligra}
\usepackage{lipsum}
\usepackage{cite}

\newcommand{\R}{\mathbb{R}}
%%%%%%%%%%%%%%%%%%%%%%%%%%%%%%%%%%%%%%%%%%%%%%%%%%%%%%
\def\BibTeX{{\rm B\kern-.05em{\sc i\kern-.025em b}\kern-.08em
    T\kern-.1667em\lower.7ex\hbox{E}\kern-.125emX}}
\begin{document}

\title{Péndulo Simple\\
{\footnotesize \scshape{Reporte 1}}
}

\author{\IEEEauthorblockN{1\textsuperscript{sd} Diego Sarceño Ramírez}
\IEEEauthorblockA{\textit{201900109} \\
}
}

\maketitle

\begin{abstract}
    
\end{abstract}

\begin{IEEEkeywords}
    
\end{IEEEkeywords}

\section{Objetivos}

\subsection{General}
    \begin{enumerate}[1.]
        \item Determinar el valor de la gravedad por medio del experimento del péndulo simple.
    \end{enumerate}
\subsection{Específicos}
    \begin{enumerate}[a)]
        \item Utilizar el tiempo de reacción para obtener medidas mejor aproximadas para el valor de la gravedad.
        \item Encontrar el número de oscilaciones apto para disminuir en cierto factor el error relativo de la gravedad.
        \item Comparar y ampliar los resultados utilizando datos obtenidos en un sofware web.
    \end{enumerate}

%\section{Introducción}
    

\section{Marco Teórico}
    
    
\section{Diseño Experimental}
    

    
\section{Resultados}
    
    
    
\section{Discución de Resultados}
    \begin{enumerate}
        \item 
        \item 
        \item 
        \item 
        \item 
    \end{enumerate}
    
\section{Conclusiones}
    \begin{enumerate}
        \item 
        \item 
        \item 
    \end{enumerate}



%\section{Recomendaciones}

\section{Anexos}
    
    
    
    
    
\begin{thebibliography}{00}
\bibitem{b1} Sears y Zemansky, \textit{Física Universitaria} 13a. Ed. México: Pearson, 2013
\end{thebibliography}

\end{document}
