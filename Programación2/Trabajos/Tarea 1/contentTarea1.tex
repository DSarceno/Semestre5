% TAREA 1 PROGRA 2






% PROBLEMA 2.1
\begin{mdframed}[style = warning]
	\begin{problem}
		Tomando el alfabeto $\Sigma = \{ a,b \}$, entonces:
			\begin{enumerate}[a)]
				\item Para los lenguajes $A = \{ a \}$ y $B = \{ b \}$, el lenguaje $A^n B$, la única palabra existente es:
					$$\boxed{a\ldots _n ab}$$
				\item Para los lenguajes $A = \{ a \}$ y $B = \{ b,\lambda \}$, el lenguaje $A^n B$, existen dos palabras, estas dadas por las dos palabras dadas en el lenguaje $B$
					$$\boxed{a\ldots _n ab \quad \quad \text{y} \quad \quad a\ldots _n a}$$
				\item Para los lenguajes $A = \{ a,\lambda \}$ y $B = \{ b \}$, el lenguaje $A^n B$, existen dos palabras, estas dadas por las dos palabras dadas en el lenguaje $A$
					$$\boxed{a\ldots _n ab \quad \quad \text{y} \quad \quad b}$$
				\item Para los lenguajes $A = \{ a,\lambda \}$ y $B = \{ b,\lambda \}$, el lenguaje $A^n B$, existen $4$ palabras
					$$
					A^n B = 
					\left\{ \begin{array}{c}
						a\ldots _n a b\\
						a\ldots _n a \\
						b \\
						\lambda
					\end{array}\right.										
					$$
			\end{enumerate}
	\end{problem}
\end{mdframed}









% PROBLEMA 2.3
\begin{mdframed}[style = warning]
	\begin{problem}
		Sabiendo que la clausula Positiva y de Klene cumplen que:
			$$L^* L = L^+$$
		Con lo que, para que se cumpla $L^* = L^+$, es necesario que $\boxed{L_\lambda = L}$.
	\end{problem}
\end{mdframed}










% PROBLEMA 2.4
\begin{mdframed}[style = warning]
	\begin{problem}
		Incisos:
			\begin{enumerate}[a)]
				\item Esta gramática es de tipo $2$, idependiente del contexto, puesto que no se genera la cadena vacía de ninguna forma, entonces solo llega a ser de tipo $2$.
				\item Esta gramática es de tipo $3$, gramática regular.
				\item Por sus claras similitudes con el inciso anterior, la gramática es de tipo $3$, puesto que igual, cumple las condiciones propuestas.
				\item Esta no llega a ser una gramática de tipo $3$ por una de las producciones de $A$, por ende, es gramática de tipo $2$, independiente del contexto. 
			\end{enumerate}
	\end{problem}
\end{mdframed}










% PROBLEMA 2.5
\begin{mdframed}[style = warning]
	\begin{problem}
		Para los lenguajes se diseño una gramática que los genera:
			\begin{enumerate}[a)]
				\item Para el lenguaje $L_1 = \{ ab^n a:n\in \N \}$, entonces, la cuartupla, se toman $\Sigma _T = \{ a,b \}$, $\Sigma _N = \{ <palabra>,<cadena>,<caracter> \}$, $S = \{ <palabra> \}$, y el conjunto de producciones:
					$$
					P = \left\{\begin{array}{ccc}
					a<palabra>a & ::= & a<cadena>a \\
					<cadena> & ::= & <caracter><cadena> \\
					& | & <caracter> \\
					<caracter> & ::= & b					
					\end{array}\right.
					$$
				\item Para el lenguaje $L_2 = \{ a^m b^n:m\geq n\geq 0 \}$, entonces, la cuartupla, se toman $\Sigma _T = \{ a,b,\lambda \}$, $\Sigma _N = \{ <palabra>,<cadena>,<caracterI>,<caracterF> \}$, $S = \{ <palabra> \}$, y el conjunto de producciones:
					$$
					P = \left\{\begin{array}{ccc}
					<palabra> & ::= & <caracterI><cadena> \\
					<cadena> & ::= & <caracterI><cadena><caracterF> \\
					& | & <caracterI> \\
					<caracterI> & ::= & a,\lambda \\
					<caracterF> & ::= & b,\lambda \\
					\end{array}\right.
					$$
				\item Para el lenguaje $L_3 = \{ a^k b^m a^n: n = k + m \}$, entonces, la cuartupla, se toman $\Sigma _T = \{ a,b,\lambda \}$, $\Sigma _N = \{ <palabra>,<cadena1>,<cadena2>,<car1>,<car2> \}$, $S = \{ <palabra> \}$, y el conjunto de producciones:
					$$
					P = \left\{\begin{array}{ccc}
					<palabra> & ::= & <cadena2> \\
					<cadena1> & ::= & <car2><cadena1><car1> \\
					& | & \lambda \\
					<cadena2> & ::= & <car1><cadena1><car1> \\
					& | & <car1><cadena2><car1> \\
					<car1> & ::= & a \\
					<car2> & ::= & b \\
					\end{array}\right.
					$$
				\item Para el lenguaje $L_4 = \{ waw^{-1}: w\in \Sigma ^* \}$, entonces, la cuartupla, se toman $\Sigma _T = \{ w,a \}$, $\Sigma _N = \{ <palabra>,,x,y \}$, $S = \{ <palabra> \}$, y el conjunto de producciones:
					$$
					P = \left\{\begin{array}{ccc}
					x<palabra>y & ::= & xay \\
					x & ::= & w \\
					y & ::= & w^{-1} \\
					\end{array}\right.
					$$
			\end{enumerate}
	\end{problem}
\end{mdframed}































