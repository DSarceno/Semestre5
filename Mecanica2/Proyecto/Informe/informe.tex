\input{Preamble_general.tex}

% COMPILAR USANDO

% pdflatex template
% bibtex template
% pdflatex template
% pdflatex template




% para los metadatos del PDF
%\usepackage[%
%bookmarksnumbered,%
%pdfauthor={Diego Sarceño (dsarceno68@gmail.com)},%
%pdftitle={Puntos de Lagrange},%
%pdfsubject={Proyecto},%
%pdfkeywords={template, template}]{hyperref}

\title{\sc Problema de los 3 Cuerpos, Puntos de Lagrange}%
\author{Diego Sarceño \thanks{Escuela de Ciencias Físicas y Matemáticas,
  Universidad de San Carlos de Guatemala.}}
\date{Guatemala, \today}
%% 20210307

\begin{document}  
\maketitle

\begin{abstract}
  \lipsum[1]
\end{abstract}


\section{Introducción}
\label{sec:intro}
\justify 
\lipsum[1]
% 
\begin{displaymath}
  \int_a^b f(x) dx,
\end{displaymath}
%
\lipsum[1]

\section{Puntos de Lagrange}
\subsection{Problema de los Tres Cuerpos}
\label{sec:Puntos de Lagrange}

El problema de los tres cuerpos no es soluble analíticamente; sin embargo, realizando ciertas restricciónes al problema llega a poderse resolver. Consideraciones a tomar:
\begin{enumerate}
	\item Dos cuerpos masivos en orbitas circulares alrededor de su centro de masa.
	\item El tercer objeto de masa $m$ tiene la condición: $m\ll M_1,M_2$, donde $M_1$ y $M_2$ son las masas de los dos objetos anteriormente mencionados.
\end{enumerate}

\subsubsection{Problema de los dos Cuerpos}
Partiendo del problema conocido de los dos cuerpos, el cual ya es conocido, se toma el sistema respecto de un punto de referencia $O$

\begin{figure}[H]
  	\centering
  	\includegraphics[scale=0.5]{Images/twoBodyProblem.png}
  	\caption{Posición de ambos cuerpos en el espacio, representando su posicion relativa y respecto a un sistema de coordenadas $O$. El vector $\vb{R}$ representa la posición del centro de masa relativa al sistema $O$. Imagen extraída de \cite{b1}, cap $4$.}
  	\label{fig:twoBodyProblem}
\end{figure}

Además, la ecuación de movimiento
\begin{displaymath}
	\mu \ddot{\vb{r}_{21}} = \vb{F}, \quad \quad \mu = \frac{M_1 M_2}{M_1 M_2},
\end{displaymath}

\noindent
al término $\mu$ se le conoce como masa reducida\footnote{De \cite{b2} capítulo $8$, sección $2$}. Para la fuerza gravitacional y una trayectoria circular, la expresión para frecuencia angular del sistema, se tiene
\begin{displaymath}
	\omega ^2 = \frac{G(M_1 + M_2)}{a^3}.
\end{displaymath}
 
\subsubsection{Sistema de Referencia Estrellado}

Para simplificar el análisis del problema, se introduce un sistema en movimiento, en concreto, en rotación. Esto es para eliminar el movimento de los cuerpos más masivos, a este nuevo sistema le llamaremos $O^*$. Este nuevo sistema tendrá su origen en el centro de masa y, a la distancia entre los cuerpos se le llamará $a$. De modo que las posiciónes en $O^*$ están dadas solo en el eje $x^*$ como
\begin{displaymath}
	x_1 = \frac{M_2}{M_1 + M_2}a, \quad x_2 = -\frac{M_1}{M_1 + M_2}a;
\end{displaymath}

\noindent
además, se fija $\vec{\omega} = \omega \vz$ y la masa $m$ solo se mueve en el plano $x^* y^*$. Con todo esto, la ecuación de moviento en el sistema $O^*$
\begin{displaymath}
	m\frac{\text{d}^{*2} r}{\text{d} t^2} = F_1 + F_2 - m\omega \cp (\omega \cp r) - 2m\omega \cp \frac{\text{d}^* r}{\text{d} t}.
\end{displaymath}

\noindent
En la cual las dos fuerzas representadas son las fuerzas de cada una de las masas sobre la masa $m$ orbitando. Expresandolas en componentes
\begin{displaymath}
	F_1 = \frac{GM_1 m}{\qty((x - x_1)^2 + y^2)^{\flatfrac{3}{2}}} (x - x_1,y), \quad \quad F_2 = \frac{GM_2 m}{\qty((x - x_2)^2 + y^2)^{\flatfrac{3}{2}}} (x - x_2,y).
\end{displaymath}


Analizando el resto de términos, se tiene que la fuerza de coriolis es perpendicular a la velocidad, por lo que no tiene un potencial asociado. Para el término de la fuerza centrífuga, es sencillo corroborar que es una fuerza central\footnote{Fuerza radial conservativa.}, lo que implica que tiene una energía potencial asociada. Desarrollando el término de la fuerza centrífuga y encontrando el potencial
\begin{align*}
	- m\omega \cp (\omega \cp r) &= m\omega ^2 \qty(x\mathbb{\vu{x} ^*} + y \mathbb{\vu{y} ^*}) \\
	V_c &= -\frac{1}{2} m\omega ^2 (x^2 + y^2).
\end{align*}

Con el análisis anterior, se concluye que la enería potencial total del sistema es
\begin{displaymath}
	V = -\frac{GM_1 m}{\qty((x - x_1)^2 + y^2)^{\flatfrac{1}{2}}} - \frac{GM_2 m}{\qty((x - x_2)^2 + y^2)^{\flatfrac{1}{2}}} - \frac{1}{2} m\omega ^2 \qty(x^2 + y^2).
\end{displaymath}

\noindent
Por simplicidad no se tomara la enería potencial, sino que el potencial gravitacional del sistema, por simplicidad al momento de desarrollar los calculos
\begin{displaymath}
	\mathcal{G} = -\frac{GM_1}{\qty((x - x_1)^2 + y^2)^{\flatfrac{1}{2}}} - \frac{GM_2}{\qty((x - x_2)^2 + y^2)^{\flatfrac{1}{2}}} - \frac{1}{2} \omega ^2 \qty(x^2 + y^2).
\end{displaymath}


\subsection{Puntos de Lagrange}
Para un sistema de dos cuerpos, los puntos de lagrange son las posiciones en donde un objeto/satélite podría estar en reposo respecto al sistema orbital. Estos puntos representan las posiciones en donde la atracción del sistema presenta una rotación sincrónica con la menor masa del sistema. Matemáticamente hablando, son las soluciones de equilibrio al problema de los $3$ cuerpos restringido. Para cualquier sistema de $3$ cuerpos existen $5$ de estos puntos de lagrange, representados por $L_1,L_2,L_3,L_4$ y $L_5$. 












\section{Implementación}
\label{sec:implementacion}

\subsection{Simplificación del Modelo}

Para que la implementación sea más simple, y que el programa/lenguaje que se utilize no gaste recursos trabajando con números de grandes ordenes de magnitud o de ordenes muy pequeños. Se realizará el siguiente cambio de variables $\xi = x/a$ y $\eta = y/a$, con lo que
\begin{displaymath}
	\xi _1 = \frac{x_1}{a}, \quad \quad \xi _2 = \frac{x_2}{a} = \xi _1 - 1,
\end{displaymath}

lo que reduce la posicion de cada masa a un número entre $0$ y $1$. Dado esto, el potencial gravitacional se reescribe a (sustituyendo $M_1$ y $M_2$ despejados de las expresiones de $\xi _1$ y $\xi _2$, esto para tener una expresión más amigables)
\begin{displaymath}
	\mathcal{G} (\xi, \eta) = \frac{G(M_1 + M_2)}{a} \qty[\frac{\xi _2}{\sqrt{(\xi - \xi _1)^2 + \eta ^2}} - \frac{\xi _1}{\sqrt{(\xi - \xi _2)^2 + \eta ^2}} - \frac{1}{2} \qty(\xi ^2 + \eta ^2)],
\end{displaymath}

la implementación se realizó en el sistema $(\xi, \eta)$.


\subsection{Implementación en \textit{Gnuplot}}

El codigo generado, al no ser un algoritmo como tal, no se mostrará el pseudocódigo, pero si se desea ver el codigo utilizado, revisar la sección de anexos \ref{sec:anexos}.

\subsubsection{Sistema Tierra$-$Luna}
Dadas las masas $M_{e} = 5.972\times 10^{24} kg$ y $M_l = 7.329\times 10^{22} kg$, se encuentran las posiciones respectivas respecto al origen del sistema estrellado en coordenadas $(\xi ,\eta)$. De modo que, la superficie potencial y sus curvas de nivel son:

\begin{figure}[H]
\centering
\subfigure[Superficie Potencial]{\includegraphics[scale=0.65]{Codigos/surfaceTierraLuna.pdf}}\qquad
\subfigure[Curvas de Nivel]{\includegraphics[scale=0.67]{Codigos/contourTierraLuna.pdf}}
\caption{Sistema Tierra$-$Luna: (a) La singularidad en la superficie representa a la tierra. Como se verá más adelante, entre mayor sea la diferencia entre las masas de los cuerpos, menor será la distinción del cuerpo de menor masa. (b) En las superficies de nivel se ven representados, no al nivel que se requiere, los puntos de interes en cierto rango.}
\label{fig: surperficie, cn t-l}
\end{figure}
\newpage
\begin{multicols}{2}
Para iniciar con algunas aproximaciones de los puntos más evidentes, se toma la coordenada $\eta = 0$. Lo que genera la siguiente curva en dicho plano: 

Lo que facilita los primeros tres puntos de lagrange estan dados en $L_1 \approx (-1.17,0)$, $L_2 \approx (-0.90,0)$ y $L_3 \approx (1,0)$. Estos puntos están representados en el sistema de coordenadas $(\xi ,\eta)$; por lo que, realizando el cambio de sistema correspondiente $x = a\xi$, donde $a$ es la distancia entre ambos cuerpos (para el sistema tierra$-$luna $a = 3.844\times 10^{8} m$), se tiene que $L_1 \approx (-4.497\times 10^{8},0)$, $L_2 \approx (-3.46\times 10^{8},0)$ y $L_3 \approx (3.844\times 10^{8},0)$, todo medido en $m$.
\columnbreak
\begin{figure}[H]
	\centering
	\includegraphics[scale=0.7]{Codigos/planePlotTierraLuna.pdf}
	\caption{Curva sobre el plano $\eta = 0$, bajo un poco de análisis de calculo diferencial, o directamente por las gráficas, se pueden representar los primeros $3$ puntos de Lagrange.}
	\label{fig:planeplot t-l}
\end{figure}
\end{multicols}





\subsubsection{Sistema Sol$-$Jupiter}




\subsubsection{Sistema Sol$-$Tierra}




\subsection{Implementación en \textit{Mathematica}}




\subsubsection{Sistema Tierra$-$Luna}





\subsubsection{Sistema Sol$-$Jupiter}




\subsubsection{Sistema Sol$-$Tierra}





\section{Conclusiones}
\label{sec:conclusiones}

\lipsum[1]

\begin{figure}[H]
  \centering
  % \includegraphics[width=.85\textwidth]{images/TrapMerr}  
  \caption{Área bajo la curva obtenido medidante las dos implementaciones,
    secuencial y paralelo en función del número de trapecios.  En el inserto
    está la diferencia del valor obtenido en la implementación secuencial
    menos el valor obtenido en la implementación paralela.  Fuente:
    elaboración propia.}
  \label{fig:Merror}
\end{figure}


\lipsum[1]

\section{Anexos}
\label{sec:anexos}

%\section*{Agradecimientos}
%\label{sec:agradecimientos}
%
%Se agradece a la ECFM-USAC por el uso del clúster Euclides donde se realizaron
%las pruebas de rendimiento reportadas en este trabajo.

% References
\nocite{*}
\bibliographystyle{IEEEannot}%
\bibliography{references}%

\begin{thebibliography}{00}
\bibitem{b1} R. Symon, \textit{Mechanics} 3a. Ed. Addison$-$Wesley Publishing Company, 1971
\bibitem{b2} R. Taylor, \textit{Classical Mechanics}, Edwards Brothers, Inc. 2005.
\end{thebibliography}

\end{document}




%%% Local Variables:
%%% mode: latex
%%% TeX-master: t
%%% End:
