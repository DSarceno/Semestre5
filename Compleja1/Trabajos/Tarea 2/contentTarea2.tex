% Tarea 1 Compleja




% PROBLEMA 1
\begin{mdframed}[style = warning]
	\begin{problem}
		Para realizar la contradicción del argumento propuesto se utilizará el ejercicio $6a$. Tomando a $f_1 (z) = z^3$ y $f_2 (z) = \sqrt{z - 1}$. Tomando el vector $z_o = 1+i$, entonces se prueban los primeros dos límites. 
			$$\lim _{z\to z_o} f_1 (z) = -2 + 2i,$$
		esto es cierto puesto que $z^3$ es entera. Además,
			$$\lim _{z\to -2 + 2i} f_2 (z) = \sqrt{-3 + 2i} = (13)^{\flatfrac{1}{4}} e^{i\flatfrac{2.55359}{2}},$$
		lo que también es cierto puesto que, la función $f_2$ es holomorfa en todos lados excepto en $\Re{z} \leq 1$ y $\Im{z} = 0$. Entonces, por el ejercicio $6a$, se sabe que la función $f_2 (f_1 (z)) = \sqrt{z^3 - 1}$, no es holomorfa en las rectas $y = \pm x$, entonces no es holomorfa en $z_o$, lo que contradice la hipótesis.
	\end{problem}
\end{mdframed}









% PROBLEMA 2
\begin{mdframed}[style = warning]
	\begin{problem}
		Incisos:
		\begin{enumerate}[a)]
			\item Tomando dos abiertos cuyas bolas (xd) son, $V_{r_1} (z_1)$ y $V_{r_2} (z_2)$. Entonces para un punto perteneciente a la intersección, se tienen dos radios, cada uno asociado a uno de los abiertos, $a$ y $a'$ respectivamente; de modo que, para un $z \in V_{r_1} (z_1) \cap V_{r_2} (z_2)$, se tiene
				$$V_a (z) \subseteq V_{r_1} (z_1)$$
			y
				$$V_{a'} (z) \subseteq V_{r_2} (z_2).$$
			Tomando $r = \text{min} (a,a')$, se tiene una bola que esta completamente contenida en la intersección de los conjuntos, i.e.
				$$z \in V_{r} (z) \subseteq V_{r_1} (z_1) \cap V_{r_2} (z_2)$$
			Lo que implica que la intersección es un abierto. Además, para demostrar que se cumple para cualquier número finito de abiertos, es simplemente hacer el proceso por pareados, lo que concluye la prueba inductiva. $\QED$
			\item Tomando un conjunto de cualquier cantidad de conjuntos abiertos $\{ C_1,\ldots \}$. Tomando un $z \in C_1$, existe $V_r (z) \subseteq C_1$ de modo que
				$$V_r (z) \subseteq C_1 \subseteq \bigcup _{i} ^{n} C_i,$$
			lo que implica que
				$$z \in \bigcup _{i} ^{n} C_i.$$
			Lo que implica que es abierto, para cualquier $n \in \N$. $\QED$
			\item Tomando el conjunto $A _n = \{ (-\flatfrac{1}{n} , \flatfrac{1}{n}): n\in \N \}$. De modo que, al tomar la intersección al infinito se tiene
				$$\bigcap _{n = 1} ^{\infty} A_n = \{ 0 \}.$$
			El cual es un conjunto cerrado.	
		\end{enumerate}
	\end{problem}
\end{mdframed}















% PROBLEMA 3
\begin{mdframed}[style = warning]
	\begin{problem}
		Para la función $f(z) = \abs{z}$, si la función es entera, implica directamente que es holomorfa. Por contradicción se asumirá la función holomorfa; entonces, la parte real e imaginaria son armónicas, de modo que:
			$$\nabla ^2 f = 0$$
		Aplicando el laplaciano:	
			$$\frac{y^2}{\qty(x^2 + y^2)^{\flatfrac{3}{2}}} + \frac{x^2}{\qty(x^2 + y^2)^{\flatfrac{3}{2}}} = 0$$
		Lo que contradice el hecho de que sea holomorfa, puesto que la igualdad no se cumple en ningún $z\in \C$, por lo tanto $f$ no es entera. $\QED$
	\end{problem}
\end{mdframed}













% PROBLEMA 4
\begin{mdframed}[style = warning]
	\begin{problem}
		$(\Rightarrow)$ Si $\pdv{f}{\bar{z}} = 0$, entonces su parte real e imaginaria es cero. Desarrollando las derivadas parciales de la definición:
			$$\pdv{f}{x} = u_x + iv_x, \quad \quad \quad \pdv{f}{y} = u_y + iv_y .$$
		Sustituyendo se tiene
			$$ = \frac{1}{2} \qty[\qty(u_x - v_y) + \qty(u_y - v_y)i] .$$
		Como la parte real e imaginaria son cero, entonces,
			$$u_x = v_y$$
			$$u_y = -v_x$$
		Las cuales son las ecuaciones de Cauchy$-$Riemann. \\
		$(\Leftarrow)$ Si cumple las ecuaciones de $CR$ entonces, 
			$$f_x = u_x + iv_x = u_x - iu_y$$
		y
			$$f_y = u_y + iv_y = u_y + iu_x.$$
		Sustituyendo en la definición dada
			$$\pdv{f}{\bar{z}} = \frac{1}{2} \qty[f_x +i f_y] = \frac{1}{2} \qty[u_x - iu_y + iu_y - u_x] = 0$$
		$\QED$
	\end{problem}
\end{mdframed}




















% PROBLEMA 5
\begin{mdframed}[style = warning]
	\begin{problem}
		Para el abierto dado, se tiene que la función es holomorfa, por lo que se cumplen las ecuaciones de $CR$; donde, $u=3$ y $v=v$
			$$u_x = v_y \quad \to \quad v_y = 0 \quad v = \phi (x)$$
			$$u_y = -v_x \quad \to \quad v_x = 0 \quad \to \quad v = \phi (y)$$
		Pero dicha constante de integración no puede ser función de ambas variables, por ende, la función $\boxed{v = c}$.
	\end{problem}
\end{mdframed}

























% PROBLEMA 6
\begin{mdframed}[style = warning]
	\begin{problem}
		Incisos:
		\begin{enumerate}[a)]
			\item Esta función no será holomorfa en donde las siguientes condiciones se cumplan: $\Re{z^3 - 1} \leq 0$ y $\Im{z^3 - 1} = 0$. Para la parte real solo se tiene
				$$x^3 - xy^2 \leq 1$$
			como condición. Para la parte imaginaría se tienen dos distintas, las cuales son:
				$$y = 0 \quad \quad \quad x^2 = y^2.$$
			De modo que, la función dada no es holomorfa en $y = 0$ y $x \leq 1$, y $x^2 - y^2$ y $0\leq 1$ (la cual no es una condición real). La segunda condición son las dos rectas perpendiculares que pasan por el origen, cuyos ángulos son $\pi / 4$ y $3\pi / 4$.
			\item Tomando la forma exponencial del seno, se tiene que:
				$$\sin{\sqrt{z}} = \frac{e^{i\sqrt{z}} - e^{-i\sqrt{z}}}{2i}$$
			Utilizando el teorema de DeMoivre se puede descomponer $\sqrt{z} = \abs{z}^{\flatfrac{1}{2}} \qty(\cos{\flatfrac{\theta}{2}} + i\sin{\flatfrac{\theta}{2}})$. Por simplicidad se tomará a $a = \abs{z}^{\flatfrac{1}{2}}$. 
				$$ = \frac{ e^{ai\coss} e^{-a\sins} - e^{-ai\coss} e^{a\sins} }{ 2i }$$			
			Sumando y restando el primer término pero con la segunda exponencial en signo positivo; de modo que:
				$$ = \frac{1}{i} e^{ai\coss} \underbrace{\qty[\frac{ e^{a\sins} + e^{-a\sins} }{ 2 }]}_{\cosh{\qty(a\sins)}} - \frac{ e^{-ai\coss} e^{a\sins} }{ 2i } - \frac{ e^{ai\coss} e^{a\sins} }{ 2i }$$
			Volviendo a sumar y restar el mismo término se tiene que:
				$$ = \cosh{\qty(a\sins)} \sin{\qty(a\coss)} -i \qty(\underbrace{\frac{ e^{ai\coss} + e^{-ai\coss} }{ 2 }}_{\cos{\qty(a\coss)}}) \cosh{\qty(a\sins)} $$
				$$ + ie^{a\sins} \qty[\underbrace{\frac{ e^{ai\coss} + e^{-ai\coss} }{ 2 }}_{\cos{\qty(a\coss)}}]$$
			Reduciendo y simplificando:
				$$ = \cosh{\qty(a\sins)} \sin{\qty(a\coss)} + i\cos{\qty(a\coss)} \qty(\underbrace{\frac{ -e^{a\sins} - e^{-a\sins} }{ 2 } + e^{a\sins}}_{\sinh{\qty(a\sins)}})$$
			El resultado de la expansión es:
				$$\sin{\sqrt{z}} = \cosh{\qty(\zsq \sins)} \sin{\qty(\zsq \coss)} + i\sinh{\qty(\zsq \sins)} \cos{\qty(\zsq \coss)}.$$
			Tomando $\theta$ como el argumento principal, es claro que dadas las funciones trigonométricas (normales e hiperbólicas), la función $\sin{\sqrt{z}}$ es holomorfa en todos los complejos excepto en $\Re{z} \leq 0$. 
		\end{enumerate}
	\end{problem}
\end{mdframed}


















% PROBLEMA 7
\begin{mdframed}[style = warning]
	\begin{problem}
		Si ambas funciones son enteras, se cumplen las ecuaciones de $CR$ en todo $\C$. Con esta idea, para $f$ se tienen las siguientes ecuaciones
			\begin{align*}
				u_x &= v_y \\
				u_y &= -v_x
			\end{align*}
		Y para $\bar{f}$, se tiene
			\begin{align*}
				U_x = V_y &\to u_x = -v_y \\
				U_y = V_x &\to u_y = v_x
			\end{align*}
		Con lo que, igualando las derivadas de la parte real, se concluye que $v = b$, lo que implica directamente que $u=a$, por ende $f(z) = a + bi$, constante. $\QED$
	\end{problem}
\end{mdframed}




















% PROBLEMA 8
\begin{mdframed}[style = warning]
	\begin{problem}
		Por contradicción, suponemos la función holomorfa, por lo que las ecuaciones de $CR$. Con lo que:
			$$v_y = u_x = e^x$$
			$$v_x = 0$$
		Integrando la primera ecuación $v = ye^x + \phi (x)$. Pero, al derivar para la segunda ecuación de $CR$, $v_x = -ye^x + \phi '(x) = 0$, lo que contradice la "utilidad" de $\phi(x)$, puesto que depende también de $y$, lo que no es posible $(\to \leftarrow)$.
	\end{problem}
\end{mdframed}





















% PROBLEMA 9
\begin{mdframed}[style = warning]
	\begin{problem}
		Como en los complejos se cumple que $z\ln{z} = \ln{z^z}$. Se puede reescribir la función como:
			$$f(z) = e^{z\ln{z}},$$
		de modo que la función $f$ es holomorfa todo $\C$ menos los reales negativos.
	\end{problem}
\end{mdframed}




























































