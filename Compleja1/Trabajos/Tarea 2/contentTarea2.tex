% Tarea 1 Compleja




% PROBLEMA 1
\begin{mdframed}[style = warning]
	\begin{problem}
		
	\end{problem}
\end{mdframed}









% PROBLEMA 2
\begin{mdframed}[style = warning]
	\begin{problem}
		
	\end{problem}
\end{mdframed}















% PROBLEMA 3
\begin{mdframed}[style = warning]
	\begin{problem}
		Para la función $f(z) = \abs{z}$, si la función es entera, implica directamente que es holomorfa. Por contradicción se asumirá la función holomorfa; entonces, la parte real e imaginaria son armónicas, de modo que:
			$$\nabla ^2 f = 0$$
		Aplicando el laplaciano:	
			$$\frac{y^2}{\qty(x^2 + y^2)^{\flatfrac{3}{2}}} + \frac{x^2}{\qty(x^2 + y^2)^{\flatfrac{3}{2}}} = 0$$
		Lo que contradice el hecho de que sea holomorfa, puesto que la igualdad no se cumple en ningún $z\in \C$, por lo tanto $f$ no es entera. $\QED$
	\end{problem}
\end{mdframed}













% PROBLEMA 4
\begin{mdframed}[style = warning]
	\begin{problem}
		
	\end{problem}
\end{mdframed}




















% PROBLEMA 5
\begin{mdframed}[style = warning]
	\begin{problem}
		
	\end{problem}
\end{mdframed}

























% PROBLEMA 6
\begin{mdframed}[style = warning]
	\begin{problem}
		Incisos:
		\begin{enumerate}[a)]
			\item 
			\item Tomando la forma exponencial del seno, se tiene que:
				$$\sin{\sqrt{z}} = \frac{e^{i\sqrt{z}} - e^{-i\sqrt{z}}}{2i}$$
			Utilizando el teorema de DeMoivre se puede descomponer $\sqrt{z} = \abs{z}^{\flatfrac{1}{2}} \qty(\cos{\flatfrac{\theta}{2}} + i\sin{\flatfrac{\theta}{2}})$. Por simplicidad se tomará a $a = \abs{z}^{\flatfrac{1}{2}}$. 
				$$ = \frac{ e^{ai\coss} e^{-a\sins} - e^{-ai\coss} e^{a\sins} }{ 2i }$$			
			Sumando y restando el primer término pero con la segunda exponencial en signo positivo; de modo que:
				$$ = \frac{1}{i} e^{ai\coss} \underbrace{\qty[\frac{ e^{a\sins} + e^{-a\sins} }{ 2 }]}_{\cosh{\qty(a\sins)}} - \frac{ e^{-ai\coss} e^{a\sins} }{ 2i } - \frac{ e^{ai\coss} e^{a\sins} }{ 2i }$$
			Volviendo a sumar y restar el mismo término se tiene que:
				$$ = \cosh{\qty(a\sins)} \sin{\qty(a\coss)} -i \qty(\underbrace{\frac{ e^{ai\coss} + e^{-ai\coss} }{ 2 }}_{\cos{\qty(a\coss)}}) \cosh{\qty(a\sins)} $$
				$$ + ie^{a\sins} \qty[\underbrace{\frac{ e^{ai\coss} + e^{-ai\coss} }{ 2 }}_{\cos{\qty(a\coss)}}]$$
			Reduciendo y simplificando:
				$$ = \cosh{\qty(a\sins)} \sin{\qty(a\coss)} + i\cos{\qty(a\coss)} \qty(\underbrace{\frac{ -e^{a\sins} - e^{-a\sins} }{ 2 } + e^{a\sins}}_{\sinh{\qty(a\sins)}})$$
			El resultado de la expansión es:
				$$\sin{\sqrt{z}} = \cosh{\qty(\zsq \sins)} \sin{\qty(\zsq \coss)} + i\sinh{\qty(\zsq \sins)} \cos{\qty(\zsq \coss)}$$
		\end{enumerate}
	\end{problem}
\end{mdframed}


















% PROBLEMA 7
\begin{mdframed}[style = warning]
	\begin{problem}
		
	\end{problem}
\end{mdframed}




















% PROBLEMA 8
\begin{mdframed}[style = warning]
	\begin{problem}
		
	\end{problem}
\end{mdframed}





















% PROBLEMA 9
\begin{mdframed}[style = warning]
	\begin{problem}
		
	\end{problem}
\end{mdframed}




























































