% Tarea 1 Compleja




% PROBLEMA 1
\begin{mdframed}[style = warning]
	\begin{problem}
		Tomando a $z = a + bi$, se tiene:
			$$\frac{a + bi}{\qty[(a + 1)^2 + bi]} = \frac{a + bi}{(a + 1)^2 - b^2 + 2bi(a + 1)} \qty(\frac{(a + 1)^2 - b^2 - 2bi(a + 1)}{(a + 1)^2 - b^2 - 2bi(a + 1)})$$
		Tomando únicamente la parte imaginaria, se tiene:
			$$b\qty[(a + 1)^2 - b^2] - 2ab(a + 1) = -2a^2 b - 2ab - b^3 + 2ab + b = -b(a^2 + b^2 - 1) = 0$$
		Lo que demuestra:
			$$\Im{\frac{z}{(z + 1)^2}}$$
		$\QED$
	\end{problem}
\end{mdframed}








% PROBLEMA 2
\begin{mdframed}[style = warning]
	\begin{problem}
		Tomando la propiedad $z\bar{z} = \abs{z} ^2$, se tiene:
			$$\abs{\frac{z + \alpha}{1 + \bar{\alpha} z}} ^2 = \qty(\frac{z + \alpha}{1 + \bar{\alpha} z}) \qty(\overline{\frac{z + \alpha}{1 + \bar{\alpha} z}})$$
		Desrrollando se tiene:
			$$ = \frac{\abs{z} ^2 + \abs{\alpha} ^2 + \bar{z}\alpha + \overline{\bar{z} \alpha}}{1 + \bar{\alpha} z + \overline{\bar{\alpha} z} + \abs{\bar{\alpha} z}} = \frac{\abs{z} ^2 + \abs{\alpha} ^2 + 2\Re{\bar{z} \alpha}}{1 + 2\Re{\bar{\alpha} z} + \abs{\bar{\alpha} z } ^2}$$
		Por las condiciones del problema se tiene que $\abs{z} ^2 \abs{\alpha} ^2 < 2$ y $\abs{\alpha z} ^2 < 1$, con lo que:
			$$< \frac{2 + 2\Re{\bar{z} \alpha}}{2 + 2\Re{\bar{\alpha} z}} = 1$$
		Asumiendo que existe caso de igualdad, se toma:
			$$\abs{z + \alpha} = \abs{1 + \bar{\alpha} z}$$
		Siguiendo la idea del inicio:
			$$(z + \alpha)(\bar{z} + \bar{\alpha}) = (1 + \bar{\alpha} z)(1 + \alpha \bar{z})$$
		Desarrollando se llega a:
			$$\abs{z} ^2 + \abs{\alpha} ^2 = 1 + \abs{z} ^2 \abs{\alpha} ^2$$
		De modo que:
			$$(1 - \abs{z} ^2)(1 - \abs{\alpha} ^2) = 0$$
		Como $\abs{\alpha} < 1$, entonces es necesario que $\abs{z} = 1$. Con esto se concluye que:
			$$\abs{\frac{z + \alpha}{1 + \bar{\alpha} z}} \leq 1$$
		Con el caso de igualdad cuando $z$ esta en el circulo unitario. $\QED$
	\end{problem}
\end{mdframed}




% PROBLEMA 3
\begin{mdframed}[style = warning]
	\begin{problem}
		Es claro que el módulo de $z$ es $1$, con lo que, la forma polar esta dada por:
			$$z = \cis {\flatfrac{\pi}{4}}$$
		Sabiendo el Teorema de Moivre, se tiene que:
			$$\cis{1^2 \flatfrac{\pi}{4}} + \cis{2^2 \flatfrac{\pi}{4}} + \cdots + \cis{12^2 \flatfrac{\pi}{4}}$$
		Con esto, para los impares se tiene que $\flatfrac{1 + i}{\sqrt{2}}$. Multiplos de $4$, el resultado es $1$ y pares no multiplos de $4$ se tiene $-1$. De modo que:
		$$z^{1^2} + z^{2^2} + \cdots + z^{12^2} = 6\frac{1+i}{\sqrt{2}}$$
		Para el lado derecho, se tiene que el inverso de $\qty(\flatfrac{1+i}{\sqrt{2}})^{-1} = \flatfrac{1-i}{\sqrt{2}}$, con esto el lado derecho de la expresión:
			$$\frac{1}{z^{1^2}} + \frac{1}{z^{2^2}} + \cdots + \frac{1}{z^{12^2}} = 6\frac{1-i}{\sqrt{2}}$$
		Multiplicando ambos resultados se tiene que: (usando $\bar{z} z = \abs{z} ^2$)
			$$\qty(z^{1^2} + z^{2^2} + \cdots + z^{12^2}) \qty(\frac{1}{z^{1^2}} + \frac{1}{z^{2^2}} + \cdots + \frac{1}{z^{12^2}}) = \boxed{36}$$
	\end{problem}
\end{mdframed}




% PROBLEMA 5
\begin{mdframed}[style = warning]
	\begin{problem}
		Sabiendo la relación de las medianas y el gravicentro, entonces tomando los segmentos de recta, simplemente se realiza dicho procedimiento. Tomando cualquier lado (en este caso se tomará $ac$) se tiene que el punto medio entre dichos vértices es:
			$$\frac{a + c}{2}$$
		Ahora, para el segmento entre este nuevo punto y el vértice restante es:
			$$b + t\qty(\frac{a + c}{2} - b)$$
		Por la relación se tiene que $t = \flatfrac{2}{3}$, con esto y desarrollando se tiene:
			$$\boxed{G = \frac{a + b + c}{3}}$$
	\end{problem}
\end{mdframed}






% PROBLEMA 6
\begin{mdframed}[style = warning]
	\begin{problem}
		Por inducción, el caso base se cumple claramente. La hipotesis inductiva: $n = k$
			$$\cos{\theta} + \cos{3\theta} + \cdots + \cos{(2k - 1)\theta} = \frac{\sin{2k\theta}}{2\sin{\theta}}$$
		Paso inductivo $k + 1$:
			$$\cos{\theta} + \cos{3\theta} + \cdots + \cos{(2k + 1)\theta} = \frac{\sin{2k\theta}}{2\sin{\theta}} + \cos{(2k + 1)\theta}$$
		Tomando el lado derecho de la ecuación, se tiene que:
			$$ = \frac{1}{2\sin{\theta}} \qty[\sin{2k\theta} + \underbrace{2\cos{\theta} \sin{\theta}}_{\sin{2\theta}} \cos{2k\theta} - 2\sin ^2 {\theta} \sin{2k\theta}]$$
			$$ = \frac{1}{2\sin{\theta}} \qty[\sin{2k\theta} \cos{2\theta} + \sin{2\theta} \cos{2k\theta}]$$
		Con lo que, utilizando identidades trigonométricas, se tiene que:
			$$\cos{\theta} + \cos{3\theta} + \cdots + \cos{(2k + 1)\theta} = \frac{\sin{2(k + 1)\theta}}{2\sin{\theta}}$$
		$\QED$
	\end{problem}
\end{mdframed}





% PROBLEMA 9
\begin{mdframed}[style = warning]
	\begin{problem}
		Como $\lambda _i \geq 0$ y $\lambda1 + \cdots + \lambda _n = 1$, esto implica que $\lambda _i \leq 1$. Por la desigualdad triangular: 
			$$\abs{\sum _i \lambda _i a_i} \leq \sum _i \abs{\lambda _i a_i}$$
		Como cada $\abs{a_i}$ y $\lambda _i$ son menores a $1$, entonces su producto también es menor a $1$, además de ser menor a ambos factores, con esto es claro que:
			$$\sum _i \abs{\lambda _i a_i} < 1$$
		Por ende:
			$$\abs{\sum _i \lambda _i a_i} < 1$$
		$\QED$
	\end{problem}
\end{mdframed}






% PROBLEMA 10
\begin{mdframed}[style = warning]
	\begin{problem}
		Expresando los complejos en forma polar y, por teorema de DeMoivre, se tiene:
			$$2^m \qty[\cos{\qty(\frac{m\pi}{3})} + i\sin{\qty(\frac{m\pi }{3})}] = 2^{\flatfrac{n}{2}} \qty[\cos{\qty(\frac{n\pi}{4})} - i\sin{\qty(\frac{n\pi}{4})}]$$
		Con esto se tiene que:
			$$\cos{\qty[\pi \qty(\frac{m}{3} + \frac{n}{4})]} + i\sin{\qty[\pi \qty(\frac{m}{3} + \frac{n}{4})]} = 2^{\flatfrac{n}{2} - m}$$
		Con esto es necesario que la parte imaginaria sea cero, por ende la suma del ángulo debe de ser par. Además, $m$ debe ser multiplo de $3$, y $n$ multiplo de $4$, también debe de cumplir que $n$ es el doble de $m$. Matemáticamente hablando:
			$$\qty(\frac{m}{3} + \frac{n}{4}) \equiv 0 \quad \text{mod} \, 2$$
		Y
			$$\frac{n}{2} = m$$
		Con esto, se tiene que:
			$$\frac{5}{6} m \equiv 0 \quad \text{mod} \, 2$$
		En otra forma $\dfrac{5}{6} m = 2q$, con lo que $m = \dfrac{12}{5} q$. Para el mínimo de $q$ para que $m$ sea entero, es decir $q = 5$, entonces $\boxed{m = 12}$ y $\boxed{n = 24}$. Al valuar se comprueba que la igualdad se cumple.
	\end{problem}
\end{mdframed}






% PROBLEMA 13
\begin{mdframed}[style = warning]
	\begin{problem}
		Sabiendo que $\abs{z} = \sqrt{2}$, de la expresión dada, se desarrolla la diferencia de cuadrados:
			$$\abs{(z^2 - 1)(z - 1)} = \abs{(z - 1)^2 (z + 1)}$$
		Como el módulo cuadrado es igual al producto entre el complejo y el conjugado:
			$$\abs{(z + 1)(z - 1)^2} ^2 = (z + 1)(\bar{z} + 1)(z - 1)^2 (\bar{z} - 1)^2$$
		Desarrollando y sustituyendo el módulo de $z$, asi como la identidad $z + \bar{z} = 2\Re{z}$, se tiene:
			$$ = \qty[9 - 4\Re{z} ^2] \qty[3 - 2\Re{z}]$$
			$$ = 27 - 18\Re{z} - 12 \Re{z} ^2 + 8\Re{z} ^3$$
		Para encontrar el máximo, el máximo de esta ecuación, también es el máximo para la expresión dada, entonces, $x = \Re{z}$, se tiene:
			$$f(x) = 27 - 18x - 12 x^2 + 8x^3$$
		Como es una función de variable real, el criterio de la primera y segunda derivada es válido, por ende:
			$$f'(x) = 4x^2 - 4x - 3 = 0$$
		De esto, las dos soluciones $x_1 = \flatfrac{3}{2}$ y $x_2 = \flatfrac{-1}{2}$. Faluandolo en la segunda derivada, se tiene que el máximo de la función esta en $x_2$. Por lo que, valuando $f(x_2)$, se tiene: $f(\flatfrac{-1}{2}) = 32$, con lo que:
			$$\boxed{\abs{(z^2 - 1)(z - 1)} = \sqrt{32}}$$
	\end{problem}
\end{mdframed}




% PROBLEMA 14
\begin{mdframed}[style = warning]
	\begin{problem}
		Incisos:
		\begin{enumerate}[a)]
			\item Sea $z\in \C$ con $\abs{\alpha} = \abs{\beta} = 1$, $\alpha , \beta \in \C$. \\
				$(\Rightarrow)$ Sabiendo que $z = \alpha + \beta$, con esto $\abs{z} = \abs{\alpha + \beta}$. Utilizando la desigualdad triangular:
					$$\abs{\alpha + \beta} \leq \abs{\alpha} + \abs{\beta} = 2$$
				Con esto se tiene que:
					$$\abs{z} \leq 2$$
				$(\Leftarrow)$ Es claro que existen $\alpha ,\beta \in \C$ tales que $\abs{\alpha} = \abs{\beta} = 1$, puesto que son números complejos pertenecientes al circulo unitario. Además, por la desigualdad triangular se tiene que $\abs{\alpha} + \abs{\beta} \geq \abs{\alpha + \beta}$. Con esto $\alpha + \beta$ es un vector dentro de la circunferencia de radio $2$. Como $\abs{z} \leq 2$, es un vector dentro del circulo de radio $2$. Como ambos son arbitrarios, sin pérdida de generalidad, $z = \alpha + \beta \QED$
		\end{enumerate}
	\end{problem}
\end{mdframed}




% PROBLEMA 15
\begin{mdframed}[style = warning]
	\begin{problem}
		Separando la expresión de la izquierda por pares, se tiene que:
			$$\underbrace{\qty(\frac{\xi}{1 + \xi ^2} + \frac{\xi ^2}{1 + \xi ^4})}_{1} + \underbrace{\qty(\frac{\xi ^3}{1 + \xi} + \frac{\xi ^4}{1 + \xi ^3})}_{2}$$
		Tomando los parentesis, tomando y desarrollando la suma de fracciones, con esto se tiene que, para $1$: (Sabiendo $\xi ^5 = 1$)
			$$\qty(\frac{\xi}{1 + \xi ^2} + \frac{\xi ^2}{1 + \xi ^4}) = \frac{1 + \xi + \xi ^2 + \xi ^4}{(1 + \xi ^2)(1 + \xi ^4)} = \frac{1 + \xi ^2 + \xi ^4 + \xi ^6}{(1 + \xi ^2)(1 + \xi ^4)} = 1$$
		Siguiendo con la parte $2$:
			$$\qty(\frac{\xi ^3}{1 + \xi} + \frac{\xi ^4}{1 + \xi ^3}) = \frac{\xi ^3 + \xi ^6 + \xi ^4 + \xi ^5}{(1 + \xi)(1 + \xi ^3)} = \frac{1 + \xi + \xi ^3 + \xi ^4}{(1 + \xi)(1 + \xi ^3)} = 1$$
		Por lo que, sumando ambas partes se tiene que:
			$$\frac{\xi}{1 + \xi ^2} + \frac{\xi ^2}{1 + \xi ^4} + \frac{\xi ^3}{1 + \xi} + \frac{\xi ^4}{1 + \xi ^3} = 2$$
		$\QED$
	\end{problem}
\end{mdframed}




% PROBLEMA 17
\begin{mdframed}[style = warning]
	\begin{problem}
		Nombrando a $\abs{z + 1} = \alpha$ y $\abs{z - 1} = \beta$, como se trata de módulos, ambos son reales. Además, nombramos a $z = a + bi$. Con esto, la igualdad se reescribe como:
			$$\sqrt{(a - \alpha) ^2 + b^2} = \sqrt{(a + \beta)^2 + b^2}$$
		Con esto es claro que:
			$$(a - \alpha) ^2 = (a + \beta) ^2$$
		Desarrollando ambos lados de la ecuación se llega a lo siguiente:
			$$(\alpha + \beta) \qty[(\alpha - \beta) - 2a] = 0$$
		Con esto se tienen dos casos:
		\begin{description}
			\item[$\alpha + \beta = 0$] Como $\alpha ,\beta \geq 0$, entonces $\alpha = -\beta = 0$. Dado esto, se tiene que:
				$$(a \pm 1) ^2 + b^2 = 0$$
			Con esto, se concluye que $\boxed{\Im{z} = 0}$ y $\boxed{\Re{z} = \pm 1}$.
			\item[$(\alpha - \beta) = 2a$] Considerando $\Re{z} = 0$, la parte imaginaria queda libre, por lo que, la otra solución a la ecuación propuesta es: $\boxed{\Re{z} = 0}$ y por lo tanto $\boxed{\Im{z} = \text{libre}}$.
		\end{description}
	$\QED$
	\end{problem}
\end{mdframed}