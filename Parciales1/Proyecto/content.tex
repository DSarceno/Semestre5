% Indice
\tableofcontents
\newpage

%% Capitulo 1
\thispagestyle{empty}
\part{Ecuaciones Diferenciales Ordinarias con más de dos Variables}

En este capítulo se estudiarán conceptos básicos para la solución de ecuaciones diferenciales parciales, para lo cual se recolectan conceptos de geometría y ecuaciones diferenciales ordinarias.

\section{Superficies y Curvas en Tres Dimensiones}

Tomando una superficie en el espacio cartesiano $(x,y,z)$
	$$f(x,y,z) = 0$$
Se escoje un punto que satisface la ecuación anterior y se incrementa $(\delta x, \delta y, \delta z)$ lo que esta relacionado con la ecuación
	$$\pdv{f}{x} \delta x + \pdv{f}{y} \delta y + \pdv{f}{z} \delta z = 0$$
En otras palabras, en la vecindad de $P(x,y,z)$ existen puntos $P'(x + \xi ,y + \eta , z + \zeta)$ que satisfacen la ecuación de superficie y que para cualesquiera dos, el tercero esta dado por:
	$$\xi \pdv{f}{x} + \eta \pdv{f}{y} + \zeta \pdv{f}{z} = 0$$
	
Las ecuaciones de la forma:
	$$x = F_1 (u,v) \quad \quad y = F_2 (u,v) \quad \quad z = F_3 (u,v)$$
Son conocidas como las \textbf{Ecuaciónes Paramétricas de la Superficie.} Esto es porque tanto $u$ como $v$ pueden ser expresadas como funciones de $x$ e $y$, una vez se encuentre dichos valores, se conocerán $u$ y $v$; además, $z$ se encuentra sustituyendo lo encontrado anteriormente en su ecuación, con esto, es claro que cumplen con la ecuación de superficie. \\

Cabe recalcar que varios conjuntos de ecucaciones paramétricas generan las mismas superficies, vease:
	$$(a\sin{u} \cos{v} , a\sin{u} \cos{v} ,a\cos{u})$$
Y 
	$$\qty(a\frac{1 - v^2}{1 + v^2} \cos{u} , a\frac{1 - v^2}{1 + v^2} \sin{u} ,\frac{2av}{1 + v^2})$$

Generan la superficie esférica
	$$x^2 + y^2 + z^2 = a^2$$