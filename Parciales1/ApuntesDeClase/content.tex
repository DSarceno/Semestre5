% CLASE 1
\label{CLASE1}

\section{Introducción}
\subsection{Distribución de Zonas}

\begin{table}[H]
	\centering
	\caption{Zona}
	\begin{tabular}{||c||c|c||}
		\hline
		\hline
		$2$ & Parciales & $40$ \\
		$n$ & Hojas de Trabajo, Investigaciones & $20$ \\
		$1$ & Proyecto & $15$ \\
		$1$ & Final & $25$ \\
		\hline
		 & Total & $100$ \\
		\hline
		\hline
	\end{tabular}
	\label{zonas}
\end{table}

\section{Origenes Físicos de las EDP's}
\subsection{Modelos de Ecuaciones Diferenciales Parciales}

Los modelos en EDP difieren claramente de los modelos EDO, puesto que la variable a entontrar depende de varias variables independientes. Físicamente, las EDPs modelan la evolución de sistemas en tiempo y espacio. \\

Ejemplos de EDPs en física:
	$$u_{tt} (x,t) = c^2 u_{xx} (x,t) \quad \quad \text{Ecuación de Onda}$$
	$$u_{xx} (x,y) + u_{yy} = 0 \quad \quad \text{Ecuación de Laplace}$$
	

Otro ejemplo claro es el flujo de calor, este modela la temperatura como una función del tiempo y de posición. La ecuación de calor es:
	$$\pdv{T}{t} = k\pdv[2]{T}{x}$$
Donde $k$ es conocida como la constante de difusividad (propiedad del material). Además, para este problema se tienen diferentes condiciones de frontera, dadas por el mismo espacio en el que se encuentra el sistema, en concreto, para este sistema se tienen:
	$$T(0,t) = T(l,t) = 0$$
Estas son distintas de las ya conocidas, condiciones iniciales, las cuales son:
	$$T(x,0) = \phi (x)$$
Para $0 < x < l$. Es claro la diferencia entre ambas condiciones, las condiciones de frontera son referidas a las variables espaciales, mientras que las condiciones iniciales son referidas al parametro temporal. A este sistema se le llama un \textbf{modelo de evolución.} \\

Los modelos que no dependen del tiempo se les conocen como \textbf{modelos estacionarios.} Por ejemplo, si $\Omega$ es un dominio espacial acotado representando una placa laminar. En la frontera de $\Omega$, denotada por $\partial \Omega$, se le impone una temperatura independiente del tiempo. Cuando la distribución de temperatura de estado estacionario $T(x,y)$ en $\Omega$ satisface la ecuación de Laplace:
	$$T_{xx} + T_{yy} = 0 \quad \quad (x,y)\in \Omega$$
Tomando la temperatura en la frontera de la lamina como $f(x,y)$, entonces se tiene:
	$$T(x,y) = f(x,y) \quad \quad (x,y) \in \partial \Omega$$
El cual es un modelo de equilibrio de temperatura. En EDP estos son modelos llamados \textbf{Problemas de Valores de Frontera.} \\

El problema de resolver la ecuación de Laplace para una región $\Omega$ con condiciones de frontera se le conoce como \textbf{El Prolema de Drichlet.} \\


% CLASE 2
\label{CLASE2}

En general, un modelo de evolución EDP de segundo orden, se tiene una ecuación de la forma:
	$$G(x,t,u,u_x,u_t,u_{xx},u_{tt},u_{xt}) = 0, \, x\in I, \, t > 0$$
Donde $I$ es un intervalo espacial acotado o no acotado. El \textbf{orden} de la ecuación es el orden de la derivada más grande que tiene; además, la \textbf{linealidad} de $G$ depende del exponente de la función y sus derivadas, la EDP es lineal si todos dichos exponentes son $1$. También, la \textbf{homogeneidad} de la EDP depende de que cada término de la ecuación contiene a $u$ o a una derivada de $u$. \\

En ciertos casos es mejor utilizar la notación de operadores, es decir:
	$$u_t - ku_{xx} = 0 \, \equiv \, Lu = 0 \quad L = \pdv{t} - k\pdv[2]{x}$$
	
Este se dice que es operador lineal si se cumple que $L(u + v) = Lu + Lv$ y $L(cu) = cLu$. Si el operador es lineal, entonces la ecuación se dice que es \textbf{homogenea}. Para las EDPs lineales las soluciones forman un espacio vectorial, es decir, las combinaciones lineales de dichas soluciones forman una solución, matemáticamente hablando, se tiene, si $\{ u_1, \ldots ,u_n \}$ son soluciones de $Lu = 0$ y $c_1,\ldots ,c_n$, entonces:
	$$c_1 u_1 + \cdots + c_n u_n$$
Es también una solución de la EDP, a esto se le conoce como \textbf{Principio de Superposición}. El mismo se extiene para sumas infinitas y para un continuo de soluciones. Es decir, si $u(x,t,\xi)$ es una familia de soluciones dependientes de un único parámetro, para toda $\xi \in J$ se puede demostrar que:
	$$\int _J c(\xi) u(x,t,\xi) \dd{\xi}$$
Es también una solución de $Lu = 0$ para condiciones específicas de $c(\xi)$. Algo importante a recalcar es la linealidad del operador en $\R$ y $\C$, es decir:
	$$L(u + iv) = Lu + iLv$$
Esto no sucede para ecuaciones no lineales, las cuales son mucho más dificil de resolver y analizar. \\

La clasificación de las EDPs puede ser dependiente de su aparición en la física o directamente en la matemática, de esto, las EDPs se clasifican:
	\begin{itemize}
		\item Ecuaciones de Onda $\equiv$ Ecuaciones Hiperbólicas
		\item Ecuación de Difusión $\equiv$ Ecuaciones Parabólicas
		\item Ecuación de Equilibrio $\equiv$ Ecuación Elíptica
	\end{itemize}
Estas clasificaciones son para EDPs lineales de segundo orden. \\

Una solución a una EDP se refiere a una función $u(x,t)$ definida para $t > 0$ y $x\in I$, que satisface por sustitución en la EDP original.

\section{Leyes de Conservación}

Muchas de las EDPs estudiadas se basan en un balance o ley de conservación. Esto es una formulación matemática que estipula que la tasa de cambio de "algo" en un dominio dado es igual a la tasa de cambio en la frontera más la tasa a la cual el "algo" es creado dentro del dominio. Un ejemplo claro de esto es el flujo a travez de tubos uniformes.


\subsection{Para tubos de área transversal}

\subsubsection{Ley Macroscópica}
Para algún instante $t$.

	$$\underbrace{\dv{t} \int _a ^b u(x,t)A\dd{x}}_{\text{Cambio en la cantidad \\ de fluido respecto de } t} = \underbrace{A\phi (a,t) - A\phi (b,t)}_{\text{Cuánto fluído entra \\ menos cuanto sale}} + \underbrace{\int _a ^b f(x,t)A\dd{x}}_{\text{Cuánto fluido se produce \\ en el segmento de tubo}}$$

\subsubsection{Ley Microscópica}

En un diferencial transversal, para un instante $t$.

	$$\underbrace{u_t (x,t) A}_{\text{Cambio en la cantidad}} + \underbrace{\phi _x (x,t) A}_{\text{Cambio en el Flujo}} = \underbrace{f(x,t) A}_{\text{Medida de "Fuentesidad"}}$$

\subsection{Método de las Características}

El modelo de advección es un modelo en el que el flujo es proporcional a su densidad, i.e.
	$$\phi = cu$$
donde $c$ es una constante. 

\begin{definicion}
	\textbf{Advección: } Es la variación de un escalar en un punto dado, por efecto de un campo vectorial.
\end{definicion}

En este caso, la ley de conservación se da en la ausencia de fuentes $f = 0$.

\begin{align}
	u_t + cu_x = 0 \label{advectionEQ}
\end{align}

A la ecuación \eqref{advectionEQ} se le llama la Ecuación de Advección. Además, es sencillo verificar que la función:
	$$u(x,t) = F(x - ct)$$
Es solución de \eqref{advectionEQ} para cualquier $F$ diferenciable. A las cuales se les llaman "\textbf{Right-traveling Waves}"\footnote{Vease Logan, "Applied PDEs", pp.15 Figure 1.4}. \\

\subsubsection{Ejemplo: Método de Características}

Tomando el problema de advección:
	$$u_t + cu_x = 0,\, x\in \R ,\, t>0$$
	$$u(x,0) = u_0 (x), \, x\in \R$$
Donde $u_0$ es una densidad inicial. Se sabe, por "Right-traveling waves" que una solución al problema esta dada por:
	$$u(x,t) = u_0 (x - ct)$$
Físicamente, la densidad inicial de la señal se mueve a una velocidad $c$, se toma la señal de densidad como una familia de curvas paralelas llamadas \textbf{características}, constantes en el espacio tiempo. Ahora, para resolver la ecuación general de advección:
	$$u_t + cu_x + au = f(x,t)$$
donde $a,c$ son constantes y $f$ una función dada. Como la ecuación se propaga a una velocidad $c$, se transforma la ecuación a un sistema de coordenadas móvil. Tomando $\xi$ y $\tau$ las nuevas variables independientes, se les llaman \textbf{Coordenadas Características}, definidas como:
	$$\xi = x - ct \quad \quad \tau = t$$
Se denota $u$ en las nuevas coordenadas como $U (\xi ,\tau) = u(\xi + c\tau ,\tau)$, por la regla de la cadena:
	$$u_t = U_\xi \xi _t + U_\tau \tau _t = -cU_\xi + U_\tau$$
Y
	$$u_x = U_\xi \xi _x + U_\tau \tau _x = U_\xi$$
Por lo que la ecuación general de advección se reduce a:
	$$U_\tau + U = F(\xi ,\tau)$$
Donde $F(\xi ,\tau) = f(\xi + c\tau ,\tau)$. Además, esta ecuación diferencial solo depende de \textbf{una} de sus variables independientes, por lo que se resuleve como una ecuación diferencial ordinaria, fácilmente por el método del factor integrante. \\

El operador de advección:
	$$L = \pdv{t} + c\pdv{x}$$
En las coordenadas características se reduce a:
	$$L_{\xi \tau} = \pdv{\tau}$$
Lo que resalta su importancia. \\
En el caso en el que la velocidad de advección sea $c(x,t)$, las coordenadas características estarán dadas por $\xi = \xi (x,t)$, $\tau = t$ donde $\xi (x,t) = C$ es la solucion general de la EDO:
	$$\dv{x}{t} = c(x,t)$$
En estas nuevas coordenadas, es claro que la EDP es transformada a una ecuación de la forma
	$$U_\tau = F(\xi ,\tau ,U)$$
En teoría, esta ecuación puede ser resuelta y luego sustituir $\xi$ y $\tau$ en términos de $x$ y $t$ para obtener $u(x,t)$.


\section{Difusión}

Para modificar la nocion de la ley de conservación, se toma un gas en un tubo. Se espera que el movimiendo randomizado y las coliciones de las moleculas causará que las concentraciones de gas se extiendan, el gas pasará de grandes a pequeñas concentraciones. De este movimiento randomizado se realizan dos observaciones:
	\begin{enumerate}[i)]
		\item El movimiento es de mayores a menores concentraciones.
		\item Cuanto mayor sea el gradiente de concentración, mayor será el flujo.
	\end{enumerate}

De esto, el flujo depende de la derivada posicional de la densidad, asumiendo una relacion lineal:
	$$\phi = -Du_x$$
con $D>0$ constante de propocionalidad, y el signo negativo es para garantizar que se cumpla la convención. A dicha ecuación se le conoce como la \textbf{Ley de Flick}, y la constante como \textbf{Constante de Difusión}. Sustituyendo esta ecuación en la Ley de Conservación, se tiene un modelo más simple:
	$$u_t - DU_{xx} = 0$$
Llamado \textbf{Ecuación de Difusión}.

\subsection{Conducción de Calor}

Considerando flujo de calor en una barra unidimensional con densidad constante $\rho$ y calor específico $C$. Aplicando la Ley de Conservación con $u$ la densidad de energía y $T$ la temperatura en $(x,t)$: 
	$$u(x,t) = \rho C T(x,t)$$
Entonces:	
	$$\rho CT_t + \phi _x = 0$$
Balance de energía cuando no hay fuentes de calor presentes, asumiendo que $\phi$ esta dado por la Ley de Flick,
	$$\phi = -KT_x$$
La cual es la \textbf{Ley de Fourier}. El calor fluye bajo el gradiente de temperatura. Con esto, la \textbf{Ecuación de Calor}
	$$T_t = kT_{xx}$$
Con $k = \flatfrac{K}{\rho C}$, llamada la constante de difusibidad.



\section{Difusión y Aleatoriedad}

\subsection{Soluciones de Estado Estable}

Las soluciones dependientes del tiempo, al momento de tener tiempos muy grandes, se llega a un estado estable de la solucion, esta separación se hace para eliminar el estado transitorio de la EDP. De modo que, las soluciones de difusión, para $t\to \infty$, se convierten en ecuaciones de Laplace. Por ejemplo:
	$$u_t - D(u_{xx} + u_{yy}) = 0$$
Tiende a convertirse en:
	$$u_{xx} = u_{yy} = 0 \quad (\text{Laplace})$$























