% CLASE 2
\label{CLASE 2}

\section{Introducción}

\subsection{Trilema de Münchhaussen}

Esto nos dice que, si se pregunta por la justificación de cierto conocimiento, y luego se pregunta por la justificación de la justificación, y así sucesivamente, hay tres posibilidades:

\begin{description}
	\item[Argumento Circular] Eventualmente se empiecen a repetir justificaciones.
	\item[Argumento \textit{ad infinitum}] Las justificaciones continúen indefinidamente.
	\item[Argumento Dogmático] Culmine con conceptos que no se ponen en duda por principio.
\end{description}

Ninguna de estas opciones es completamente satisfactoria.  Cada una de ellas tiene ventajas y desventajas.

\section{Combinatoria}
\subsection{Principio de la Multiplicación}

Si cierta tarea puede realizarse de $m$ formas distintas, y para cada una de ellas, una segunda tarea se puede realizar de $n$ formas diferentes, entonces ambas tareas se pueden realizar de $mn$ formas distintas.  \textit{Esto es, asumiendo que las tareas deban realizarse en el orden en que fueron mencionadas.}